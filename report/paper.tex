\documentclass{sig-alternate-br}
\usepackage{multirow}
\usepackage{rotating}
\usepackage{xcolor,colortbl}
\usepackage[official]{eurosym}
\usepackage{hyperref}

\begin{document}

\CopyrightYear{2013} 

\title{Two Approached for Graphing E-Science Data making use of Resolution Flexibility}

\numberofauthors{3}
\author{
\alignauthor Niek Tax\\
       \affaddr{University of Twente}\\
       \affaddr{P.O. Box 217, 7500AE Enschede}\\
       \affaddr{The Netherlands}\\
\alignauthor Bas Janssen\\
       \affaddr{University of Twente}\\
       \affaddr{P.O. Box 217, 7500AE Enschede}\\
       \affaddr{The Netherlands}\\
\alignauthor David Huistra\\
              \affaddr{University of Twente}\\
              \affaddr{P.O. Box 217, 7500AE Enschede}\\
              \affaddr{The Netherlands}\\
}

\maketitle
\begin{abstract}
When representing sensor network data of a large period in a graph, a lot of data is aggregated resulting in slow response times. To counter this issue, pre-aggregated
datasets are used. This reports discusses solutions to several issues concerned with making the pre-aggregated datasets more useful, such as learning from user statistics to calculate the optimal pre-aggregation levels.
\end{abstract} 

\keywords{Pre-aggregated datasets, Integer Linear Programming (ILP), User statistics}

\section{Introduction}
Hier komt de introduction!
\section{Problem Statement}
The problem is defined as follows: When representing this data in a graph a maximum resolution of points is encountered that is often smaller then the amount of data points. To solve this issue, the entire dataset is separated in a maximum resolution amount of blocks, and of each of these blocks the average is calculated and shown as a data point in the graph. The amount of dataset values that is combined into a single data point in the graph is called the factor. When users of the graph request data over a large period, this will result in a high amount of data being requested from the the database that in turn results in a high amount of loading time. This causes the average response time to plot a graph to become unacceptable high.\\

%the average value of a number of values is calculated and this average is shown as a datapoint in the graph.

% The only way the data can be accessed is by a graph which plots about 600 points and therefore, a GROUP BY statement is introduced to collect the data needed for the graph. Because the users need different domains and different group factors, all calculations are done on the fly, which is recourse heavy and can take up a lot of time and is %therefore unacceptable.\\

We are provided with the complete dataset of sensor data, running from 31st of December 2006 to the 7th of august 2011. Using this dataset, we have been given the task to use pre-aggregations to speed up the response time in the process of plotting the graph.\\

There are several preconditions to solve this problem:
\begin{itemize}
\item maintain usage statistics 
\item re-write queries to use pre-aggregation levels
\item re-organize pre-aggregation levels
\item use the graph user interface provided.
\end{itemize}

After reading the already provided paper "Graphing of E-Science Data with varying user requirements", we came up with several sub problems.\\

\textbf{Problem 1: The user statistics have to be collected.}\\
By using the graph, the user is providing us with important information about the requested data (the desirable domain and aggregation factors).\\

\textbf{Problem 2: We require a set of pre-aggregated datasets at the start of the program.}\\
These datasets can be used to serve the user with fast response time from the beginning without any user statistics.

\textbf{Problem 3: We require a way to determine what pre-aggregated dataset can be used to serve a users request.}

%Problem 2: We are unable to predict the domain and aggregation factors from the beginning.
%Starting from scratch, the desired parameters are unknown. Still, it is unacceptable to provide the user with a slow and recourse heavy system.\\

\textbf{Problem 4: We need a tool to calculate the optimal aggregation factors.}\\
After the user statistics are known, a process has to be found to translate these statistics to useful new pre-aggregations. Also, some users would like a more precise dataset in their graph compared to other users. These users are willing to give in on speed.\\

\textbf{Problem 5: The offset problem needs to be addressed.}\\
As described in the paper\cite{wombacher2011}, the pre-aggregations are based on not only an factor but also an offset. For example: if the domain of the dataset is 0-11 and your pre-aggregations are 0-2, 3-5, 6-8, 9-11 (factor = 3) Then if you try to access the domain 4-7 (factor = 3 as well), then you cannot use 3-5 and 6-8 directly, because these include data which is not in the domain 4-7. 

\textbf{Problem 6: The graphical user interface needs some bugfixing and some extra features.}\\
Next to some additional features to provide a better way of accessing the wanted data, we have been given the task to address the bugs present in the GUI.

\textbf{Problem 7: Optimizing the inevitable load operations.}

\section{Solutions}
\subsection{Problem 1: The user statistics have to be collected}
On the moment the SQL-statement is being constructed to query the database, the aggregation factor and domain are known. At that moment, these parameters are logged for user statistics.
There are two tables which hold information about the queries. The first table: query log holds information about every parameter the user is able to affect. Each query results in a new row containing the start point, end point and factor used by the graph and a timestamp when the query was executed. The second table: factor log holds information about the factors used only: when a query contains a new factor which is not documented set in this table, a new row is appended containing the new aggregation factor. Else the count related to the factor in incremented by one.\\

\subsection{Problem 2: Determining the basic aggregation levels}
\subsubsection{Precise method}
To provide a fast response time from the beginning, a set of pre-aggregations is to be calculated based on the expected queries of users. Wombacher and Aly \cite{wombacher2011} showed the Integer Linear Programming (ILP) approach to be the most promising solution to the problem of selecting pre-aggregation factors in a study in which multiple solutions have been reviewed. The Integer Linear Program formulated by Wombacher and Aly \cite{wombacher2011} did not take into account the flexibility of plotting at different resolutions (in our case: 300-600 data points). To be able to use Integer Linear Programming to find an optimal set of pre-aggregation factors we propose a small modification to the Integer Linear Program formulated by Wombacher and Aly \cite{wombacher2011} as follows.
\paragraph{A Integer Linear Program in a multi-resolution setting}
To the Integer Linear Program definition given by Wombacher and Aly \cite{wombacher2011}, we redefine the query cost definition $L_{i,j}$ such that it equals the cost for plotting query i using factor j at the resolution resulting in the cheapest cost for this query and factor. The following variable definitions are added or revised:
\begin{itemize}
\item R is the set of allowed resolutions
\item $C_{i,j,r}$ is the cost for executing query $q_{i}$ with pre-aggregation level $A_{j}$ using plotting resolution $R_{r}$; given that the factors are defined on the highest possible resolution, the cost is defined as $\frac{f_{i}*\frac{max(R)}{r}}{f_{i}^a}$ if $f_{i}*\frac{max(R)}{r}$ mod $f_{j}^a = 0$, or a high constant otherwise indicating that the resolution adjusted query factor $f_{i}*\frac{max(R)}{r}$ is not an integer multiple of the pre-aggregate $f_{j}^a$.
\item $L_{i,j}$ is the minimal cost for executing query $q_{i}$ with pre-aggregation level $A_{j}$; the cost is the minimum of the set of costs $C_{i,j,r}$ with $r \in R$.
\end{itemize}

\paragraph{Choosing the query input}
With no usage statistics available at the beginning it is desirable to optimize the ILP for as many queries as possible. In a study by Meindl and Templ \cite{meindl2012}, the LPsolve solver which Wombacher and Aly \cite{wombacher2011} used to solve the Integer Linear Program performed 19 times slower and was able to find an optimal solution in more than 40 times fewer problems than the best performing ILP-solver included in the study (Gurobi \cite{gurobi}). As the risk of not finding an optimal solution (within feasible time) increases with an increasing number of variables, which is also stated by Wombacher and Aly \cite{wombacher2011}, the use of a more efficient and effective ILP-solver will contribute to our ability of calculating the optimal pre-aggregation factors for larger query sets.\\

A test run showed it to be infeasible to solve the ILP for all queries theoretically possible, as the solver did not terminate within three days time. The question rises which queries are most important to optimize. A query of factor x worst case needs to retrieve x data items (in case no pre-aggregation fits and the non-aggregated data is used). Therefore we can conclude that the larger the query factor, the larger the worst case cost of the query. Our solution is based on factors outputted by the ILP-solver for the 50\% of the theoretically possible queries with the largest factor. 

\subsection{Problem 3: Determining the correct aggregation level to use}
\subsubsection{Fast method}
The standard query simply selects all database entries between two points of time and will aggregate the result set until there are only 600 or fewer (minimum of 300) points left. Given a set of pre-aggregation tables, we can simply make the standard query use a pre-aggregated dataset instead of the entire database. To determine what pre-aggregation table to use in the query, the following method is executed. Check for all aggregation factors: Is it there an aggregation factor where the following expression is true 0.5 * aggregationFactor < requiredAggregationFactor < aggregationFactor, then the aggregation dataset with this factor should be used, as it will result between 300 - 600 points without requiring run-time aggregation. Otherwise, find largest pre-aggregation factor in the set that is smaller then the required aggregation factor. Using this aggregations dataset will result in the least amount of run-time aggregations to provide between 300-600 aggregations.

%TODO: hoe bepaald de precise method welke pre-aggregated dataset gebruikt word voor de user query?

While this method is not very precise, it does however make sure a minimum of run-time aggregations are done, making the system fast. We believe that for new users, this is a good way to get acquainted with the dataset.

\subsubsection{Precise method}

\subsection{Problem 4: We need a tool to calculate the optimal aggregation factors}
\subsection{Problem 5: The offset problem needs to be addressed}
\subsection{Problem 6: The graphical user interface needs some bugfixing and some extra features}
From the beginning of the project, we were designated to use the jFreeChart library and the included files to update the graph using the new data points. The task description mentioned already that the GUI still contained some errors and our goal was to fix them.

The first error we came across was that the JdbcYIntervalSeries' update method needed a data conversion to handle the input from the database. The epoch time in the database is defined in seconds, but the epoch time needed in the graph must be defined in milliseconds. To handle this problem, we updated all the methods in JdbcYIntervalSeries by multiplying the time with 1000 and dividing the time variables before the query with 1000.

The second error we came across was that the domain scrollbar left side could surpass his right side. Therefore, the extent became zero and an exception was thrown by jFreeChart. To fix this bug, we updated the adjustmentValueChanged() function in GraphGui class. It now checks if the extent evaluates to zero and updates this variable with a constant factor if so.

Apart from the errors, we also included some extra features in the GUI. The horizontal scrollbar, used for updating the domain of the graph, was not implemented very well. When an user zooms in and reaches (for example) the domain that 600 points would represent a day, the "to-right"-button and the "go-left"-botton would shift the domain with several days. Therefore, navigating to the right or left on this level of detail would be very inconvenient.
To solve this inconvenience, we introduced two buttons who first look at the level of detail and then calculate to which domain the graph should be shifted. If the scrollbar is beginning at point x and the extent of the scrollbar is y, the button "to-right" lets the domain start from x+(y/2) and uses the same extent as before the shifting. Of course, the "to-left" button uses x-(y/2) to determine the start of the domain in the graph.
%
% The following two commands are all you need in the
% initial runs of your .tex file to
% produce the bibliography for the citations in your paper.

\bibliographystyle{abbrv}
\bibliography{paper}

\balancecolumns

\onecolumn
\appendix

\section{TODO}
Hier komen misschien dingen in een appendix

\end{document}
